% Options for packages loaded elsewhere
\PassOptionsToPackage{unicode}{hyperref}
\PassOptionsToPackage{hyphens}{url}
%
\documentclass[
]{article}
\author{}
\date{\vspace{-2.5em}}

\usepackage{amsmath,amssymb}
\usepackage{lmodern}
\usepackage{iftex}
\ifPDFTeX
  \usepackage[T1]{fontenc}
  \usepackage[utf8]{inputenc}
  \usepackage{textcomp} % provide euro and other symbols
\else % if luatex or xetex
  \usepackage{unicode-math}
  \defaultfontfeatures{Scale=MatchLowercase}
  \defaultfontfeatures[\rmfamily]{Ligatures=TeX,Scale=1}
\fi
% Use upquote if available, for straight quotes in verbatim environments
\IfFileExists{upquote.sty}{\usepackage{upquote}}{}
\IfFileExists{microtype.sty}{% use microtype if available
  \usepackage[]{microtype}
  \UseMicrotypeSet[protrusion]{basicmath} % disable protrusion for tt fonts
}{}
\makeatletter
\@ifundefined{KOMAClassName}{% if non-KOMA class
  \IfFileExists{parskip.sty}{%
    \usepackage{parskip}
  }{% else
    \setlength{\parindent}{0pt}
    \setlength{\parskip}{6pt plus 2pt minus 1pt}}
}{% if KOMA class
  \KOMAoptions{parskip=half}}
\makeatother
\usepackage{xcolor}
\IfFileExists{xurl.sty}{\usepackage{xurl}}{} % add URL line breaks if available
\IfFileExists{bookmark.sty}{\usepackage{bookmark}}{\usepackage{hyperref}}
\hypersetup{
  hidelinks,
  pdfcreator={LaTeX via pandoc}}
\urlstyle{same} % disable monospaced font for URLs
\usepackage[margin=1in]{geometry}
\usepackage{graphicx}
\makeatletter
\def\maxwidth{\ifdim\Gin@nat@width>\linewidth\linewidth\else\Gin@nat@width\fi}
\def\maxheight{\ifdim\Gin@nat@height>\textheight\textheight\else\Gin@nat@height\fi}
\makeatother
% Scale images if necessary, so that they will not overflow the page
% margins by default, and it is still possible to overwrite the defaults
% using explicit options in \includegraphics[width, height, ...]{}
\setkeys{Gin}{width=\maxwidth,height=\maxheight,keepaspectratio}
% Set default figure placement to htbp
\makeatletter
\def\fps@figure{htbp}
\makeatother
\setlength{\emergencystretch}{3em} % prevent overfull lines
\providecommand{\tightlist}{%
  \setlength{\itemsep}{0pt}\setlength{\parskip}{0pt}}
\setcounter{secnumdepth}{-\maxdimen} % remove section numbering
\ifLuaTeX
  \usepackage{selnolig}  % disable illegal ligatures
\fi

\begin{document}

\hypertarget{material-and-methods}{%
\section{Material and Methods}\label{material-and-methods}}

\hypertarget{our-data-sets}{%
\subsection{Our data sets}\label{our-data-sets}}

For the analysis four data sets were provided.

The first data set is a Gene expression data frame. The Gene expression
data frame contains 60,000 genes and and their expression in 10,000
patients. It is derived from The Cancer Genome Atlas (TCGA). The
expression of the genes was obtained by RNA-seq, a sequencing method to
evaluate the activity of the genes \ref{RNAseq}.

The second data frame contains 37 clinical annotations like Tumor type,
age, gender, etc. concerning the 10,000 patients from the Gene
expression data frame.

The third object contains 5 lists for the focused analysis, one for each
tumor type (BRCA, KIRC, LUAD, PRAD, THCA). For the focused analysis the
THCA data was used. The THCA data contains 3 data frames, each one with
information about the same 60 patients. The first one with gene
expression data from THCA tissue, the second one from normal tissue and
the third one contains the clinical annotations like age and gender.
Gene expression data was obtained by RNA-seq.

The fourth object contains 46 pathways involved in phenotypes partly
included in the hallmarks of cancer and the genes involved in those
pathways.

SIND DIE DATEN NORMALISIERT ODER ALS COUNTS?

\hypertarget{metabolic-pathway-selection}{%
\subsection{Metabolic pathway
selection}\label{metabolic-pathway-selection}}

Furthermore metabolic pathways had to be selected from the Molecular
Signature Database (MSigDB) (QUELLE) to compare them with the given
Hallmark-Pathways and identify more pathways, that are important for the
development of cancer. Therefore it was important, that as many genes
from the selected pathways as possible are also included in the provided
Hallmark pathways. To identify the relevant pathways, the intersection
of genes was calculated and the genes with an intersection of at least
99\% were maintained for further analysis.

To avoid duplicates in between the metabolic pathways and between the
Hallmark pathways and the metabolic pathways, the pathways were checked
for duplicates with the Jaccard index. Pathways with a sum of Jaccard
indices beyond the 1-sigma range were discarded.

\hypertarget{preprocessing}{%
\subsection{Preprocessing}\label{preprocessing}}

\hypertarget{deleting-nas}{%
\subsubsection{Deleting NAs}\label{deleting-nas}}

Deleting of NAs was done with the R-function na.omit(x).

\hypertarget{low-variance-filtering}{%
\subsubsection{Low-variance filtering}\label{low-variance-filtering}}

Low variance filtering is performed to delete genes with a low variance
from the data set. To calculate the variance the r-function var(x) is
used and genes with a lower variance than a certain threshold value are
removed.

For focused analysis the variance of each gene in tumor tissue was
calculated and genes with a low variance were deleted from the data sets
for tumor and normal tissue.

\hypertarget{biotype-filtering}{%
\subsubsection{Biotype filtering}\label{biotype-filtering}}

For biotype filtering the biotype of each gene was determined (protein
coding, RNA,\ldots) and compared with the biotypes of other data sets.
To allow an appropriate comparison of the expression data, only biotypes
were kept, that are available in a big amount in all of the data sets.
The biotype can be determined with the R-function checkbiotypes(x) from
the package biomaRt.

\hypertarget{selection-of-metabolic-pathways}{%
\paragraph{Selection of metabolic
pathways}\label{selection-of-metabolic-pathways}}

da eine hohe jaccard summe eine hohe überschneidung mit anderen pathways
bedeutet. In einer heatmap sind hohe Jacccard indices weiß bis rot
gefährbt. Ein niedriger Jaccard index ist blau gefärbt.

The test for duplicates in between the selected metabolic pathways and
between the hallmark pathways and the metabolic pathways the Jaccard
index and its sum were calculated.

There were a few duplicates between the metabolic and Hallmark pathways.
Those metabolic pathways with a great sum of Jaccard indices were
discarded. The success of cleaning was checked with another heatmap and
new Jaccard indices. There are only a few elements with another color
than blue. It can be assumed, that the selection of relevant pathways
was successful. The number of metabolic pathways could be reduced from
xxx to 600.

\hypertarget{descriptive-analysis}{%
\subsection{Descriptive analysis}\label{descriptive-analysis}}

\hypertarget{mean-variance-plot}{%
\subsubsection{Mean-variance plot}\label{mean-variance-plot}}

In a mean-variance plot the variance is plotted against the mean of
expression values of the single genes. Therefore, the variance and mean
were calculated by the R-function var(x) and mean(x).

\hypertarget{violin-plot}{%
\subsubsection{Violin plot}\label{violin-plot}}

To check the distribution of a data set and compare it with other data
sets violin plots are used. Violin plots are tilted and mirrored density
plots of gene expression values. The y-axis shows the gene expression
value and the x-axis shows the amount of genes with a certain gene
expression value.

\hypertarget{volcano-plot}{%
\paragraph{Volcano plot}\label{volcano-plot}}

A volcano plot is used to identify significantly differentially
expressed genes. In the following analysis a volcano plot is used to
identify which genes are significantly diferentially expressed in the
analysed THCA tissue, compared to the analysed normal tissue. Therefore
the mean of each gene is calculated for normal and THCA tissue and used
for the calculation of the Log2-Foldchange (Log2FC) in the following
way:

\[
log2FC = mean(normal tissue) - mean(tumor tissue)
\] In the next step, a two-sided t test was performed to determine the
significance of a difference in expression.

To avoid the accumulation of type 1 errors, a bonferroni correction was
performed. n is the number of genes in the cleaned data set for focused
analysis:

\[
A \alpha = \frac{0.025}{n}
\]

In the volcano plot the -log10 of the calculated p values is plotted
against die Log2FC. Genes with a a lower p-value than the corrected
alpha-value are significantly differentially expressed. If the Log2FC is
additionally higher than 0.1, the genes are significantly over expressed
in tumor tissue, if the Log2FC is higher lower than -0.1, the genes are
significantly under expressed in tumor tissue.

\hypertarget{comparing-of-pathways}{%
\subsection{Comparing of Pathways}\label{comparing-of-pathways}}

After reducing the number of Genes of the Gene Expression data frame and
the THCA data by data cleaning, a descriptive analysis was performed
with a Mean-variance plot and five violin plots of the TGCA data frame.
The descriptive analysis of the THCA data was performed with a volcano
plot and the distribution of the Tumor-specific data was displayed with
violin plots.

\hypertarget{pca}{%
\subsubsection{PCA}\label{pca}}

The package xxx is used to perform the PCA. Therefore the data obtained
from the GSEA was used. After performing the PCA, the results were
plotted to visualize the different clusters.

The PCA was performed for pathway and gene activity. For analysis of the
gen activity the package xxx was used.

Dazu wurde noch analysiert, wie die Pathways auf die PCs verteilt sind.

\hypertarget{tcga-data}{%
\subsection{TCGA data}\label{tcga-data}}

What kind of data do we have? \#\# Used Packages

show a table!

\end{document}
